% IEEE conference paper on MCP security
\documentclass[conference,a4paper]{IEEEtran}
\IEEEoverridecommandlockouts

\usepackage[turkish]{babel}
\usepackage[utf8]{inputenc}
\usepackage[T1]{fontenc}
\usepackage[pdftex]{graphicx}
\usepackage{multirow}
\usepackage{cite}
\usepackage[cmex10]{amsmath}
\usepackage{siunitx}
\usepackage{array}
\usepackage[caption=false,lofdepth,lotdepth]{subfig}
\usepackage{acronym}
\usepackage{xcolor}
\usepackage{microtype}
\usepackage{booktabs}

\hyphenation{op-tical net-works semi-conduc-tor}
\setlength{\textfloatsep}{5pt}

\AtBeginDocument{\renewcommand\tablename{TABLO}}

% Türkçe özet ve anahtar kelime ortamlarını IEEE ortamlarına eşleştir
\renewenvironment{ozet}{%
  \renewcommand\abstractname{Özet}%
  \begin{abstract}%
}{%
  \end{abstract}%
}
\renewenvironment{IEEEanahtar}{%
  \renewcommand\IEEEkeywordsname{Anahtar Kelimeler}%
  \begin{IEEEkeywords}%
}{%
  \end{IEEEkeywords}%
}

\begin{document}

\title{Model Bağlam Protokolü (MCP) Ekosisteminin Eleştirel Bir Güvenlik İncelemesi\\
       A Critical Security Review of the Model Context Protocol (MCP) Ecosystem}

\author{%
  \IEEEauthorblockN{Yusuf Talha ARABACI}%
  \IEEEauthorblockA{%
    Yazılım Mühendisliği Yüksek Lisans Öğrencisi\\%
    Karabük Üniversitesi\\%
    Karabük, Türkiye%
  }%
}

\maketitle

\begin{ozet}
2024 yılı sonlarında tanıtılan açık bir standart olan Model Bağlam Protokolü (Model Context Protocol, MCP), büyük dil modellerinin (LLM) harici araçlar ve veri kaynakları ile etkileşimini standartlaştırmayı hedeflemektedir. Bu çalışma, hızla gelişen MCP ekosistemini mimari, güvenlik ve ajanik yönetişim perspektiflerinden eleştirel bir yaklaşımla sentezlemektedir. LLM entegrasyon sürecinde karşılaşılan $N \times M$ parçalanma sorununa çözüm sunmayı amaçlayan MCP, kendisini yapay zekâ alanı için bir nevi evrensel bağlayıcı (USB-C) olarak konumlandırmaktadır. 2025'te eş zamanlı yayımlanan çok sayıda akademik yayın, bu protokolün, birinci nesil ajan sistemlerinin ölçeklenebilirlik ve güvenlik sınırlarına karşı geliştirilmiş acil bir endüstriyel yanıt olduğunu göstermektedir. Makalede, MCP'nin temel mimari bileşenleri (İstemci-Sunucu, Araçlar, Kaynaklar), Hou ve ark. (2025) tarafından önerilen on altı senaryolu tehdit taksonomisi ve Hasan ve ark. (2025) tarafından ampirik olarak saptanan güvenlik ve bakım (maintainability) zorlukları analiz edilmektedir. Son olarak, protokol seviyesindeki savunma mekanizmaları (MCP-Guard, zMCP) ile merkeziyetsiz bir ekosistemde sürdürülebilir büyümeyi güvence altına almak için kritik olan ajanik güvenlik yönetişimi tartışılmaktadır.
\end{ozet}
\begin{IEEEanahtar}
Model Bağlam Protokolü, MCP, mimari paradigmalar, tehdit taksonomisi, ajanik güvenlik yönetişimi.
\end{IEEEanahtar}

\begin{abstract}
The Model Context Protocol (MCP) is an open protocol introduced in late 2024 to standardize the interaction of large language models (LLMs) with external tools and data sources. This paper critically synthesizes the rapidly growing MCP ecosystem from architectural, security, and agentic governance perspectives. MCP aims to solve the $N \times M$ integration problem in LLM tooling, positioning itself as a ``universal connector'' (USB‑C) for AI. The burst of MCP-related publications in 2025 suggests that the protocol is an urgent industrial response to the scalability and security limits of first-generation agent systems. We analyze MCP's core architectural components (Client--Server, Tools, Resources), the sixteen-scenario threat taxonomy proposed by Hou et al.\ (2025), and the empirically observed security and maintainability challenges identified by Hasan et al.\ (2025). Finally, we discuss protocol-level defense mechanisms (MCP‑Guard, zMCP) and the agentic security governance required for sustainable growth in a decentralized MCP ecosystem.
\end{abstract}
\begin{IEEEkeywords}
Model Context Protocol, MCP, security, architecture, threat taxonomy, agentic governance.
\end{IEEEkeywords}

\IEEEpeerreviewmaketitle

\section{Giriş: Ajan Paradigması ve Entegrasyon Zorunluluğu}
Büyük dil modelleri (LLM'ler), statik bilgiye dayalı metin üreten sistemler olmaktan çıkıp, gerçek dünya görevlerini yerine getirebilen otonom ajanlara evrilmektedir. Bu paradigma değişimi, LLM'lerin harici API'ler, veritabanları ve dosya sistemleri gibi kaynakları güvenilir ve ölçeklenebilir bir biçimde kullanmasını zorunlu kılmaktadır. Geleneksel yaklaşımda, her bir LLM'nin (N) her harici araçla (M) bütünleşmesi için özel bağlayıcılar gerekmesi, ``$N \times M$ entegrasyon sorunu'' olarak bilinen kayda değer bir karmaşıklığa yol açmaktadır.

Model Bağlam Protokolü (MCP), bu parçalı yapıya karşı, birleşik, çift yönlü iletişimi ve dinamik keşfi mümkün kılan açık bir standart sunarak çözüm getirmektedir. MCP, LLM'lerin yalnızca araçların yüzeysel açıklamalarına dayandığı basit işlev çağrılarından, protokol odaklı bir bağlam sunumuna geçişi sağlayan mimari bir dönüşümü ifade etmektedir. Anthropic tarafından 2024 sonlarında tanıtılan bu protokol, kısa sürede fiili bir endüstri standardı haline gelmiş ve 2025 itibarıyla OpenAI, Google DeepMind ve Microsoft gibi öncü teknoloji firmalarının ürünlerinde de desteklenmeye başlamıştır \cite{hou2025landscape,singh2025survey}.

Bu bildirinin amacı, MCP ekosistemine dair güncel literatürü bütüncül bir yaklaşımla ele alarak mimari paradigmaları, tehdit sınıflandırmasını ve ajanik güvenlik yönetişimini eleştirel bir çerçevede incelemektir. Bu doğrultuda, çalışmanın temel katkıları şu şekilde özetlenmektedir:
\begin{itemize}
  \item MCP mimarisini, tehdit modelini ve ampirik bulguları tek bir çerçevede sentezleyerek ekosistemin güçlü ve zayıf yönlerini karşılaştırmalı olarak tartışmak.
  \item Literatürde dağınık halde bulunan kıyaslama (benchmark), stres testi ve kırmızı takım (red teaming) çalışmalarının sonuçlarını, MCP'ye özgü siber riskleri görünür kılmak amacıyla tematik olarak sınıflandırmak.
  \item Protokol, sunucu (host) ve organizasyon düzeylerinde uygulanabilir savunma stratejileri ile MCP tedarik zinciri hijyenine ve ajanik güvenlik yönetişimine ilişkin eyleme geçirilebilir öneriler sunmak.
\end{itemize}

\section{Mimari Paradigmalar ve Temeller}
MCP, ajanların harici sistemlerle güvenli ve standart bir biçimde etkileşim kurmasını sağlayan bir istemci-sunucu mimarisine dayanır. Bu bölümde, mimarinin temel bileşenleri, protokol yapısı ve sunucu yaşam döngüsü ayrıntılı olarak incelenmektedir.

\subsection{Çekirdek Bileşenler ve Protokol Yapısı}
MCP mimarisi üç temel bileşenden oluşur: LLM'i barındıran \emph{MCP Host}, sunucularla protokol üzerinden iletişim kuran \emph{MCP Client} ve harici yetenekleri standart bir formatta sunan \emph{MCP Server}.

Protokolün temel özellikleri şu şekilde özetlenebilir:
\begin{itemize}
  \item \textbf{İletişim Standardı:} MCP, uygulama tutarlılığını sağlamak amacıyla JSON-RPC 2.0 standardını temel alır. Bu tercih, istek/yanıt döngülerinin net bir şekilde tanımlanmasını, hata kodlarının standartlaşmasını ve mesaj düzeyinde kimlik doğrulama ile yetkilendirme katmanlarının eklenmesini kolaylaştırır.
  \item \textbf{Temel Primitifler:} MCP, dış dünyadan sağlanan bağlamı üç ana soyutlama altında toplar:
  \begin{itemize}
    \item \emph{Araçlar (Tools):} LLM tarafından çağrılabilen, yürütülebilir fonksiyonlardır (örneğin, dosya okuma, HTTP isteği gönderme).
    \item \emph{Kaynaklar (Resources):} LLM'nin okuyabildiği pasif veri varlıklarıdır (dosyalar, veritabanı satırları, log kayıtları vb.).
    \item \emph{İstemler (Prompts):} Yeniden kullanılabilir ve standartlaştırılmış iş akışlarını veya etkileşim kalıplarını içeren şablonlardır.
  \end{itemize}
\end{itemize}

Bu soyutlamalar, LLM'nin akıl yürütme ortamı ile harici yürütme ortamı arasında belirgin bir güven sınırı tanımlar. Host tarafında model, araç ve kaynak açıklamalarını kullanarak planlama yaparken, sunucu tarafında ise gerçek sistem çağrılarının gerçekleştirildiği kod yürütme katmanı yer alır \cite{krishnan2025multiagent}.

\begin{figure}[!t]
  \centering
  {\shorthandoff{=}\includegraphics[width=\columnwidth]{protocol}}
  \caption{MCP istemci-sunucu mimarisinin temel akış şeması. Şema, MCP Host/Client/Server bileşenlerini, JSON-RPC tabanlı iletişimi, araçlar ve kaynaklar üzerinden bağlam akışını ve $N \times M$ entegrasyon sorununa getirilen standartlaştırılmış çözümü göstermektedir.}
  \label{fig:mcp-architecture}
\end{figure}

\subsection{Sunucu Yaşam Döngüsü ve Yönetimi}
MCP entegrasyonu, yalnızca bir arayüz tanımı sunmaktan ibaret olmayıp, bütüncül bir yaşam döngüsü yönetimi gerektirir. Hou ve ark. \cite{hou2025landscape}, bir MCP sunucusunun yaşam döngüsünü \emph{oluşturma}, \emph{dağıtım}, \emph{işletme} ve \emph{bakım} olmak üzere dört aşamaya ayırarak bu aşamaları on altı temel faaliyete bölmüştür.

Oluşturma aşamasında araç açıklamaları ve şemalar tasarlanırken, dağıtım aşamasında sunucunun hangi ortamda, hangi kimlik bilgileriyle ve hangi sanal alan (sandbox) profiliyle çalışacağı belirlenir. İşletme ve bakım aşamaları ise loglama, versiyonlama, zafiyet yönetimi ve geri alma (rollback) gibi stratejileri kapsar. Özellikle dağıtım aşaması, sunucunun güvenli ve tutarlı çalışmasını belirlediği için kritik bir öneme sahiptir; zira hatalı yapılandırmalar veya aşırı yetkilendirilmiş çalışma ortamları ciddi saldırı yüzeyleri oluşturabilir.

\section{Tehdit Taksonomisi ve Güvenlik Zorlukları}
MCP, LLM'leri harici yürütme ortamlarına bağlayarak önemli yeni saldırı yüzeylerini beraberinde getirmektedir. Bu bölümde, Hou ve ark. tarafından geliştirilen tehdit taksonomisi ve ekosistemin karşılaştığı temel güvenlik zorlukları özetlenmektedir.

\subsection{Tehdit Taksonomisi ve Saldırgan Modelleri}
Hou ve ark. \cite{hou2025landscape}, MCP sunucu yaşam döngüsü boyunca ortaya çıkan güvenlik ve gizlilik risklerini dört ana saldırgan türüne göre sınıflandırmaktadır:
\begin{enumerate}
  \item \textbf{Kötü niyetli geliştiriciler:} Arka kapı (backdoor) içeren veya kasıtlı olarak zayıf yapılandırılmış sunucular yayımlayan aktörler.
  \item \textbf{Harici saldırganlar:} Ağ trafiğini dinleyerek, zayıf kimlik doğrulama mekanizmalarından veya hatalı TLS yapılandırmalarından faydalananlar.
  \item \textbf{Kötü niyetli kullanıcılar:} LLM'yi manipüle ederek, sahip olmadıkları yetkileri dolaylı yoldan kötüye kullanan son kullanıcılar.
  \item \textbf{Güvenlik açıkları:} İstem dışı kodlama hataları, yanlış yapılandırmalar ve güncel olmayan bağımlılıklardan kaynaklanan zafiyetler.
\end{enumerate}

Bu taksonomi, sunucu yaşam döngüsünün her aşamasında ortaya çıkabilecek on altı farklı tehdit senaryosunu içermekte ve MCP ekosistemi için bir referans risk haritası sunmaktadır.

\subsection{Kritik Saldırı Vektörleri}
MCP'ye özgü en kritik saldırı vektörleri şu şekilde özetlenebilir:
\begin{itemize}
  \item \textbf{Araç zehirlenmesi (tool poisoning):} Saldırganın, bir aracın tanımına (description) gizli ve zararlı talimatlar enjekte ederek LLM'nin akıl yürütme sürecini manipüle etmesidir. Bu saldırı türü, modelin anlamsal yoruma olan güvenini istismar eder ve geleneksel güvenlik duvarlarının "anlamı" tarayamaması nedeniyle kendine özgü bir zorluk teşkil eder \cite{xing2025guard}.
  \item \textbf{Dolaylı istem enjeksiyonu (indirect prompt injection):} Saldırganın, LLM'nin okuyacağı ancak doğrudan güvenmediği bir MCP kaynağına (resource) kötü niyetli talimatlar yerleştirmesiyle gerçekleştirilir. Model, bu zehirli kaynağı güvenilir bir bağlam gibi yorumlayarak kritik bir aracı tetiklemeye ikna edilebilir.
  \item \textbf{Araç zincirleme suistimali (tool chaining abuse -- STAC):} Tek başlarına zararsız görünen düşük riskli araçların, LLM tarafından birleştirilerek veri sızdırma veya yetkisiz eylemler gibi yüksek etkili sonuçlar doğurmasıdır. Bu saldırılar, modelin kümülatif zararı öngörme ve planlama konusundaki yetersizliklerini istismar eder.
\end{itemize}

\subsection{Ampirik Güvenlik ve Kod Kalitesi Analizleri}
Hasan ve ark. \cite{hasan2025firstglance}, 1.899 açık kaynak MCP sunucusunu inceledikleri geniş ölçekli bir yazılım mühendisliği çalışmasında, ekosistemin güvenlik hijyeni açısından endişe verici bir tablo çizdiğini göstermiştir. Çalışmanın bulgularına göre:
\begin{itemize}
  \item İncelenen sunucuların yaklaşık \%7,2'sinde genel güvenlik açıkları bulunmaktadır.
  \item Sunucuların \%5,5'inde MCP'ye özgü araç zehirlenmesi riskleri raporlanmıştır.
  \item Projelerin \%66'sında, uzun vadeli bakımı zorlaştıran yazılım mühendisliği kusurları (``code smells'') tespit edilmiştir.
\end{itemize}

Bu bulgular, geliştiricilerin protokole hızla adapte olurken güvenlik konusundaki en iyi uygulamaları ihmal ettiğini ve MCP'ye özgü zafiyet tarama tekniklerine acil bir ihtiyaç duyulduğunu göstermektedir.

\section{Ajanik Güvenlik Yönetişimi ve Savunma Çerçeveleri}
MCP'nin getirdiği güvenlik zorlukları, savunma stratejilerinin yalnızca çıktı filtrelemenin ötesine geçerek, planlama ve yürütme denetimine odaklanan bir ajanik güvenlik yönetişimi yaklaşımıyla ele alınmasını gerektirmektedir.

\subsection{Protokol Düzeyinde Güvenlik ve Yönetişim}
MCP sunucularının genellikle bireysel geliştiriciler veya küçük ekipler tarafından bağımsız olarak yönetilmesi, merkezi bir denetim mekanizmasının yokluğunda yama tutarsızlıklarına ve yapılandırma farklılıklarına neden olmaktadır. Gelecekteki çalışmaların, ekosistem genelinde dayanıklılığı artırmak için zorunlu yapılandırma doğrulaması, otomatik sürüm kontrolü ve bütünlük denetimi gibi teknik yönetişim çözümlerine odaklanması gerektiği vurgulanmaktadır.

Kurumsal düzeyde benimseme için Sıfır Güven (Zero Trust) ilkeleri büyük önem taşımaktadır. zMCP gibi önerilen protokol uzantıları, her işlem için kimlik doğrulama ve yetkilendirmeyi zorunlu kılarak Tam Zamanında (Just-In-Time, JIT) erişim kontrolü stratejilerini desteklemektedir. Bu yaklaşım, olası bir ihlal durumunda saldırganın sistem içindeki hareket kabiliyetini radikal biçimde kısıtlar.

Dağıtım aşamasında her MCP sunucusunun özel bir sanal alan (sandbox) ortamında çalıştırılması, dosya sistemi, ağ ve sistem komutları gibi kaynaklara erişimlerinin en az ayrıcalık ilkesine göre kısıtlanması kritik bir gerekliliktir. Bu izolasyon, olası bir komut enjeksiyonu veya uzaktan kod çalıştırma (RCE) saldırısı durumunda zararın kontrol altında tutulmasını sağlar.

\subsection{İleri Savunma Çerçeveleri}
Xing ve ark. tarafından önerilen MCP-Guard çerçevesi \cite{xing2025guard}, araç zehirlemesi ve veri sızdırma saldırılarına karşı katmanlı bir savunma mimarisi sunmaktadır. Bu mimari, hafif statik tarama, derin öğrenme tabanlı zararlı istem dedektörü ve son karar için hafif bir LLM "hakem" modülünü içeren üç aşamalı bir huni yapısıyla çalışarak anlamsal saldırıları yüksek doğrulukla tespit edebilmektedir.

MCPSafetyScanner gibi ajan tabanlı denetim araçları, MCP sunucularının güvenlik açıklarını otomatik olarak değerlendirmeye yönelik ilk adımları temsil etmektedir. Ayrıca, bilgi akışı kontrolü (Information Flow Control, IFC) prensipleri, zehirli bilginin kritik kararları etkilemesini önleyerek veri sızıntısı ve dolaylı istem enjeksiyonu gibi tehditlere karşı en umut verici mimari yaklaşımlardan biri olarak kabul edilmektedir.

\section{Ampirik Değerlendirme, Uygulamalar ve Standardizasyon}

\subsection{Performans Değerlendirmesi ve Sınırlamalar}
MCP'nin etkinliği, büyük ölçüde LLM'lerin araç kullanımındaki ampirik performansıyla ölçülmektedir. Bu alanda iki temel kıyaslama (benchmark) çerçevesi öne çıkmaktadır:
\begin{itemize}
  \item \textbf{MCPGAUGE:} Song ve ark. \cite{song2025help}, LLM-MCP etkileşimlerini proaktiflik, uyumluluk, etkinlik ve ek yük (overhead) olmak üzere dört boyutta değerlendiren MCPGAUGE çerçevesini sunmuştur. Çalışmaları, MCP entegrasyonunun otomatik bir performans artışı sağlamadığını ve LLM'lerin özellikle uyumluluk ile proaktiflik konularında belirgin sınırlamalara sahip olduğunu göstermektedir.
  \item \textbf{MCP-Universe:} Luo ve ark. \cite{luo2025universe}, gerçek dünya MCP sunucularıyla etkileşimli görev setleri sunarak en gelişmiş (frontier) modellerin bile bu gerçekçi senaryolarda ciddi performans kısıtları yaşadığını ortaya koymuştur.
\end{itemize}

LiveMCP-101 \cite{yin2025livemcp} ve MCPToolBench++ \cite{fan2025mcptoolbench} gibi diğer çalışmalar da MCP ekosisteminin stres altındaki davranışını ölçmekte; zamanlama, hata yayılımı ve araç başarısızlıklarının sistemin genel performansına etkilerini nicel olarak analiz etmektedir.

\subsection{Alan Uzmanlığı ve Standardizasyon Örnekleri}
MCP'nin soyut yapısı, farklı alanlara kolayca uyarlanabilir olduğunu göstermektedir:
\begin{itemize}
  \item \textbf{Biyoinformatik (MCPmed):} MCPmed topluluk girişimi, GEO ve STRING gibi geleneksel ve insan odaklı web sunucularını, LLM'ler için makine tarafından işlenebilir bir katmana dönüştürmeyi önermektedir. Bu girişim, FAIR (Bulunabilir, Erişilebilir, Birlikte Çalışabilir, Yeniden Kullanılabilir) ilkelerinin yapay zekâ sistemlerine nasıl uygulanabileceğini de tartışmaktadır \cite{flotho2025mcpmed}.
  \item \textbf{Uyarlanabilir ulaşım sistemleri:} Chhetri ve ark. \cite{chhetri2025transport}, MCP benzeri bağlam protokollerinin akıllı ulaşım altyapılarında kullanımı için bir mimari çerçeve sunmakta ve bu alanda güvenlik (security) ile emniyet (safety) gereksinimlerinin bir arada ele alınması gerektiğini vurgulamaktadır.
  \item \textbf{Kritik altyapı ve ekonomi:} Kritik altyapılarda varlık keşfi \cite{coppolino2025asset} ve iktisadi araştırmalara yönelik ajan tabanlı yaklaşımlar \cite{korinek2025agents}, MCP tarzı protokollerin alan-odaklı veri kaynaklarını standartlaştırarak ajanlara nasıl açılabileceğini göstermektedir.
\end{itemize}

Bu örnekler, MCP'nin yalnızca teknik bir entegrasyon standardı olmanın ötesinde, alanlar arası bir güvenlik ve yönetişim sorunu olarak da değerlendirilmesi gerektiğini ortaya koymaktadır.

\section{Sonuç ve Gelecek Yönelimleri}
Model Bağlam Protokolü (MCP), LLM tabanlı ajan sistemlerinde standart bağlam paylaşımı ve birlikte çalışabilirliği sağlayarak $N \times M$ entegrasyon sorununa önemli bir çözüm sunmaktadır. Ancak protokolün hızla benimsenmesi, beraberinde geniş ve paylaşılan bir saldırı yüzeyi de yaratmıştır. Mevcut akademik çalışmalar, araç zehirlenmesi, dolaylı istem enjeksiyonu ve araç zincirleme suistimali gibi yeni tehditlerin ciddiyetini ve MCP sunucularının güvenlik hijyeni konusundaki eksikliklerini açıkça göstermektedir.

Gelecekteki araştırma ve geliştirme çabalarının aşağıdaki alanlara odaklanması gerektiği düşünülmektedir:
\begin{itemize}
  \item \textbf{Güven sınırlarının güçlendirilmesi:} Merkeziyetsiz bir ekosistemde güvenilirliği sağlamak amacıyla zorunlu yapılandırma doğrulaması, otomatik sürüm kontrolü ve bütünlük denetimi gibi teknik yönetişim mekanizmalarının uygulanması.
  \item \textbf{Protokol düzeyinde güvenlik:} zMCP gibi Sıfır Güven (Zero Trust) yaklaşımlarının protokole entegre edilmesi ve araç zehirlenmesi gibi anlamsal saldırıları engelleyebilecek bilgi akışı kontrolü (IFC) mekanizmalarının geliştirilmesi.
  \item \textbf{LLM optimizasyonu:} MCPGAUGE tarafından tespit edilen uyumluluk ve ek yük (overhead) sorunlarını gidermek amacıyla, LLM'lerin MCP kullanımı için özel olarak eğitilmesi ve kod yürütme (code execution) gibi verimli kalıpların yaygınlaştırılması.
  \item \textbf{Çoklu ajan koordinasyonu:} Birden fazla ajanın aynı MCP sunucusunu eş zamanlı kullandığı senaryolar için kilitlenme (deadlock) önleme ve kaynak paylaşım protokollerinin tasarlanması.
\end{itemize}

MCP, sürdürülebilir bir büyüme için gerekli temel yapı taşlarını sunsa da, potansiyelini tam anlamıyla gerçekleştirebilmesi, yenilikçilik ile güvenlik arasındaki hassas dengenin kurulmasına ve ajanik güvenlik yönetişiminin ekosistemin merkezine yerleştirilmesine bağlıdır.

\begin{thebibliography}{99}
\bibitem{hou2025landscape} X.~Hou, Y.~Zhao, S.~Wang, and H.~Wang, ``Model Context Protocol (MCP): Landscape, Security Threats, and Future Research Directions,'' \emph{arXiv preprint arXiv:2503.23278}, 2025.
\bibitem{krishnan2025multiagent} N.~Krishnan, ``Advancing Multi-Agent Systems Through Model Context Protocol: Architecture, Implementation, and Applications,'' \emph{arXiv preprint arXiv:2504.21030}, 2025.
\bibitem{ehtesham2025survey} A.~Ehtesham, A.~Singh, G.~K.~Gupta, and S.~Kumar, ``A Survey of Agent Interoperability Protocols: Model Context Protocol (MCP), Agent Communication Protocol (ACP), Agent-to-Agent Protocol (A2A), and Agent Network Protocol (ANP),'' \emph{arXiv preprint arXiv:2505.02279}, 2025.
\bibitem{hasan2025firstglance} M.~M.~Hasan, H.~Li, E.~Fallahzadeh, G.~K.~Rajbahadur, B.~Adams, and A.~E.~Hassan, ``Model Context Protocol (MCP) at First Glance: Studying the Security and Maintainability of MCP Servers,'' \emph{arXiv preprint arXiv:2506.13538}, 2025.
\bibitem{flotho2025mcpmed} M.~Flotho, I.~F.~Diks, P.~Flotho, L.-A.~G.~Molano, P.~Hirsch, and A.~Keller, ``MCPmed: A Call for MCP-Enabled Bioinformatics Web Services for LLM-Driven Discovery,'' \emph{arXiv preprint arXiv:2507.08055}, 2025.
\bibitem{mastouri2025rest} M.~Mastouri, E.~Ksontini, and W.~Kessentini, ``Making REST APIs Agent-Ready: From OpenAPI to MCP Servers for Tool-Augmented LLMs,'' \emph{arXiv preprint arXiv:2507.16044}, 2025.
\bibitem{fan2025mcptoolbench} S.~Fan, X.~Ding, L.~Zhang, and L.~Mo, ``MCPToolBench++: A Large Scale AI Agent Model Context Protocol MCP Tool Use Benchmark,'' \emph{arXiv preprint arXiv:2508.07575}, 2025.
\bibitem{xing2025guard} W.~Xing, Z.~Qi, Y.~Qin, Y.~Li, C.~Chang, J.~Yu, C.~Lin, Z.~Xie, and M.~Han, ``MCP-Guard: A Defense Framework for Model Context Protocol Integrity in Large Language Model Applications,'' \emph{arXiv preprint arXiv:2508.10991}, 2025.
\bibitem{song2025help} W.~Song, H.~Zhong, Z.~Ding, J.~Xue, and Y.~Li, ``Help or Hurdle? Rethinking Model Context Protocol-Augmented Large Language Models,'' \emph{arXiv preprint arXiv:2508.12566}, 2025.
\bibitem{luo2025universe} Z.~Luo, Z.~Shen, W.~Yang, Z.~Zhao, P.~Jwalapuram \emph{et al.}, ``MCP-Universe: Benchmarking Large Language Models with Real-World Model Context Protocol Servers,'' \emph{arXiv preprint arXiv:2508.14704}, 2025.
\bibitem{yin2025livemcp} M.~Yin, D.~Shen, S.~Xu, J.~Han, S.~Dong \emph{et al.}, ``LiveMCP-101: Stress Testing and Diagnosing MCP-enabled Agents on Challenging Queries,'' \emph{arXiv preprint arXiv:2508.15760}, 2025.
\bibitem{chhetri2025transport} G.~Chhetri, S.~Somvanshi, M.~M.~Islam, S.~Brotee, M.~S.~Mimi \emph{et al.}, ``Model Context Protocols in Adaptive Transport Systems: A Survey,'' \emph{arXiv preprint arXiv:2508.19239}, 2025.
\bibitem{tokal2025agentx} S.~S.~K.~A.~Tokal, V.~Jha, A.~Eswaran, P.~Jayachandran, and Y.~Simmhan, ``AgentX: Towards Orchestrating Robust Agentic Workflow Patterns with FaaS-hosted MCP Services,'' \emph{arXiv preprint arXiv:2509.07595}, 2025.
\bibitem{he2025automated} P.~He, C.~Li, B.~Zhao, T.~Du, and S.~Ji, ``Automatic Red Teaming LLM-Based Agents with Model Context Protocol Tools,'' \emph{arXiv preprint arXiv:2509.21011}, 2025.
\bibitem{singh2025survey} A.~Singh, A.~Ehtesham, S.~Kumar, and T.~T.~Khoei, ``A Survey of the Model Context Protocol (MCP): Standardizing Context to Enhance Large Language Models (LLMs),'' \emph{Preprints 202504.0245}, 2025.
\bibitem{bhandarwar2025integrating} N.~Bhandarwar, ``Integrating Generative AI and Model Context Protocol (MCP) with Applied Machine Learning for Advanced Agentic AI Systems,'' \emph{International Journal of Computer Trends and Technology}, 2025.
\bibitem{coppolino2025asset} L.~Coppolino, A.~Iannaccone, R.~Nardone, and A.~Petruolo, ``Asset Discovery in Critical Infrastructures: An LLM-Based Approach,'' \emph{Electronics}, vol.~14, no.~32, p.~3267, 2025.
\bibitem{korinek2025agents} A.~Korinek, ``AI Agents for Economic Research,'' NBER Working Paper~34202, 2025.
\end{thebibliography}

\end{document}
